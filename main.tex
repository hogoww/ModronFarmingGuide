\documentclass{article}
\usepackage[utf8]{inputenc}
\usepackage{booktabs}
\usepackage{hyperref}

\title{Idle Modron Briv Farming}
\author{Hogo}
\date{July 2022}

\begin{document}


\maketitle
\tableofcontents

\newcommand{\separatingLine}{  
    \addlinespace[2pt]
    \hline
    \addlinespace[2pt]
}



%add champion inlined image https://tex.stackexchange.com/questions/374192/how-to-use-figures-as-inline-images
\newcommand{\artemis}{Artemis}
\newcommand{\arkhan}{Arkhan}
\newcommand{\azaka}{Azaka}
\newcommand{\blackViper}{Black Viper}
\newcommand{\briv}{Briv}
\newcommand{\baeloth}{Baeloth}
\newcommand{\desmond}{Desmond}
\newcommand{\dragonbait}{Dragonbait}
\newcommand{\minsc}{Minsc}
\newcommand{\nayeli}{Nayeli}
\newcommand{\nerys}{Nerys}
\newcommand{\shandie}{Shandie}
\newcommand{\tyril}{Tyril}
\newcommand{\zorbu}{Zorbu}

\section{Disclaimer}

This is a work in progress.
I work on it when I find the time and or feel like it.
You may notice that there are a lot of placeholders for now.
I'm updating it at my own rhythm.
Particularly, I am yet to add pictures.
This is planned, it's just easier to add a lot of text first.

If you have remarks on things that are \textbf{already} typed in, please feel free to comment on it in the modron briv thread on the Idle champion discord.
(links will be added someday).
Or wisp me on Discord directly: hogo\#8547.
If you're polite and nice, I usually am nice too :)

\section{Introduction}

Using Modron briv farming is an advanced technique use to idle gem farm in Idle champion more efficiently.
This guide is an improvement on the work of Zeke et al (todoref) hopefully clearing some things up, and adding some knowledge over this technique that we learn by discussing this on the discord (todoref).

It looks to use \textbf{\briv} which is the single most powerful speed champion in the game at this time.
Briv is able to jump over one or many areas thanks to his unnatural haste ability.

This technique is \textbf{scriptless} and only uses the game mechanics defined by developers.
Compared to script gem farming we have
\begin{itemize}
    \item More constraints (Only one formation, no script clicks, no offline stacking [...]);
    \item More lag (Because we require more champions, cannot clear memory leak with resets [...]);
    \item Less profitability;
    \item more set up difficulties. 
\end{itemize}
Then why use it you may ask?
I've seen some people say it's because they don't like scripting, some people have issue with the scripting tools, and many other reasons.
Mine is that I like the challenge of setting this up !
%add inlined image https://tex.stackexchange.com/questions/374192/how-to-use-figures-as-inline-images


This is not a plug and play guide.
It is heavily dependent on your favors, champions, gears and multiple others.
You need to understand it in order to set it up.
I only give you the most important information here, but you have to do understand it in order to set it up.
If you do not understand it, it means this guide may be improved, so let me know !
Moreover, it might be a bit intimidating at first.
But don't loose hope !
Once you get it, you'll be able to set up anywhere you'll like !


\section{Definition}

I start here by giving you some definitions that will ease the comprehension of the technique.
Particularly, I set the definitions of multiple Idle Champion defined concepts as well as adding a few of ours.

\subsection{Idle Champion Reminder}

\paragraph{Campaign.}
A campaign is a group of multiple adventure.
There are permanent campaigns and non permanent ones such as events todoref Wiki campaigns.

todoexample

\paragraph{Adventure.}

An adventure is a part of a campaign.
They are selected from the world map.
It also contains variants.

todoexample

\paragraph{Area.}

An \textbf{Area} is a map on which your character fight.
It's part of an adventure.
It's a zone between 1 and 50.
For example Area one is represented in \textbf{Zone}s 1, 51, 101, 151 [...].

todo example
\paragraph{Zones.}
A \textbf{Zone} is the current stage your party is fighting.
It's a zone multiplied by fifty.
A \textbf{Zone} uses an area and adds scaling information on the area.
Talking about a Zone is often done by using the "z" character before a number.
z51 = Zone 51.

\paragraph{Base Ultimate Damage (BUD).}

BUD is the highest damage that was set by a champion.
It is used to compute the damages of champion's ultimate damages, and firebreath potions.
More importantly it is the best visualization of your current damages.

\subsection{Guide specific.}

\subsubsection{Concepts}

\paragraph{Perfect Jumps.}

\briv chances to jump over the next N areas.
This chances are regulated by \briv 4th item slot.
Perfect jumps are when \briv has 100\% chance of skipping the next N area.
The item level required to be perfect jumps are computable with multiple tools.
Particularly I personnaly use the great Byteglow online tool todoref.
This technique is significantly easier to use with a perfect jump. 


\paragraph{Click Wall.}

Click damage is upgradable.
Your click damage is therefore able to kill monsters.
It even one shot monsters with high enough upgrades in it !
We call \textbf{Click Wall} the zone in which the click damage is not able to one shot monsters anymore.
Click damages might still be to clear your \textbf{Click Wall} zone, but it's still your \textbf{Click Wall}.

It is roughly calculated by taking the power of your favor (enable scientific notation) and multiply it by 7.43.
For example, I have 5e100, my click wall is around zone $100 * 7.43 = 743$ !

\paragraph{Stacking Zone (= Hard Wall).}

The \textbf{Stacking Zone} is the zone in which neither the click damage nor the champions are able to kill anything.
In regular play, it's the maximum zone reachable with all the tricks you are able to think of.
In this guide, it's a zone on which your click damage are not able to kill anything anymore, in spite of the click wall extender champion (Section~\ref{sec:clickWallExtender}).

\paragraph{Zone Spread.}

The \textbf{Zone Spread} is the number of areas between your Click Wall and your Hard Wall.
It allows you to loose click damage until a point in which the click damage is not able to kill anything anymore.
%Need to find correct credit.
Calculated by NNN, you need a \textbf{Zone Spread} of at least 17 areas to loose enough click damage to not kill anything on your hard wall.

\subsubsection{Champion Jobs}

\paragraph{Click Wall Extender.}
\label{sec:clickWallExtender}

The \textbf{Click Wall Extender} is a character (or more) that increases the click damages.
Having a \textbf{Click Wall Extender} allows to go from the Click Wall to the Hard Wall as fast as possible !
There are multiple champions possible for the jobs.
Particularly, \minsc is allows click damage to be increased against a kinds of enemies out of 5.

todolist

\paragraph{Trigger Champion.}

A \textbf{Trigger Champion} is a champion that gives a boost of damage from some event.
In Zeke's guide for example, it's \dragonbait.
When the creep enrage, dragonbait provides an additional damage buff.

\paragraph{BUD Setter.}

The \textbf{BUD Setter} is the champion that hits the hardest and sets the BUD.
It's usually the most buffed damage dealer.


\section{Using Modron Briv Stacking}

\subsection{Requirements ?}

Requierements:
\begin{itemize}
    \item A modron core with automation available;
    \item A \briv with at least 1.2 jumps (ref zeke);
    \item A trigger champion;
    \item Adequate favor;
    \item Adequate adventure.
\end{itemize}

Optional (but pretty nice !)
\begin{itemize}
    \item A Click Wall Extender
\end{itemize}
 


\subsection{General Concept}

Beware: this technique is \textbf{significantly more difficult} to set up for non-perfect jump.
Therefore I describe the case of non-perfect jump)
I advise to stay at perfect jumps starting at 3 jumps.
Only put contracts when you have enough to go the next perfect jump chance.
Moreover, if you are using a non-perfect jump \briv, I'd advise to look at the part for perfect jumps first, as it is probably easier to understand.

The gem farm goes about as such:
\begin{itemize}
    \item Clicker kill everything until the click wall;
    \item Clicker or champion kill mobs until the stacking zone (Section~\ref{sec:stepOne});
    \item The party arrives to the stacking zone nobody kills anything(Section~\ref{sec:stepTwo});
    \item \briv tanks and re-stacks for the next run (Section~\ref{sec:stepThree});
    \item When \briv has enough stacks, the party kills everything thanks to the trigger champion (Section~\ref{sec:stepFour});
    \item Restart (Section~\ref{sec:stepFive})!
\end{itemize}

Each of these steps is described further in the following.

\subsubsection{Step One: choose Your Target}
\label{sec:stepOne}

Any map can be a target.
For example, I once set this technique up to farm \briv chests during his event.
However, the ideal properties are a map on which you are able to use a Click Wall Extender.
Usually, multiple maps in a row need to have a certain property.

\begin{description}
    \item[Minsc.] (Slot 7) \minsc adds damages against a certain kind of enemy with the ability \textit{Favored Enemy}.
    Moreover, it also adds damage to the Click damage !
    Possible types are: Human, Beast, Undead, Fey and Monstrosities.
    Moreover this is buffed by \minsc 4th item.
    Therefore it adds a lot more damage than others champions against those targets !
    This is the most used champion to use as a click wall extender.
    This is also the only one represented in the table, because it's the one that got the most feedback on, and the one I personally use.
    \item[Zorbu] (Slot 12) \zorbu's \textit{Lifelong Enemy} increases the damages against Humain, Beast, Undead and Drow.
    However this does \textbf{NOT} apply on click damages, nor does it apply on other champions.
    Therefore using \zorbu as a click wall extender does not work.
    However, he does more damages against this kind of enemies.
    Therefore, if he is the BUD setter, he smooths the transitions toward the stacking area in the same manner a click wall extender does.
    \item[Azaka] \azaka has a debuff that increases the champions damages when fitting outdoor.
    Therefore the transition from the click wall to the stacking zone may work if you find an adventure on which your jumps get from outdoor to undoor, with the required zone spread.
    \item[Nerys] \nerys has a small debuff against undeads that does not scale, but seem to apply on click damage.
    This can help if you have a small zone spread.
\end{description}

Others may be a good fit, but are not as good as the ones described here.

The stacking area should be a non boss area.
Range have been shown to give more stacks for the same amount of time and damages in the maps we used.


\subsubsection{Step Two: Tune Click Damage}
\label{sec:stepTwo}

To tune your click damage, you have to tweak the gold find.
The main source of gold find is favor.
I always recommend to shoot just bellow the point that you need, to be able to tune it with other methods.
Particularly, a level 15 speed core provides up to e4.5 gold find //tocheck.
To compute the required favor, divide the target area with the magic number 7.43.
For example, I want to stack at z691 on Tall Tales.
For that purpose, I look up in the Table \ref{tbl:adventures}, the click wall should be at the lowest at 656.
$656 / 7.43 = 88$, therefore I need to farm e88 favor.
As I said before, I tend to aim a bit lower to be able to tweak it if needed.
In this case, I would go to e85 and add modron nodes.

To get more Gold find, you may:
\begin{itemize}
    \item Farm favor;
    \item Add modron gold find nodes;
    \item Champions.
    The main champion you may use are Rust, Jarlaxle and Azaka.
    They are the most consistant gold find providers.
    Particularly Rust, because he won't get more item level from opening silver and gold chests.
\end{itemize}

Note that the click damage should be at least e3, ideally even e4 lower than the stacking zone monster's health.

\subsubsection{Step Three: Tune Briv's Stacking}
\label{sec:stepThree}

To get stacks, \briv must tank attacks from enemies.
You therefore have to clear the area only after you get enough stacks, otherwise you will not be able to jump all the way back to your stacking zone.


Once you have a stacking zone in mind, you may compute how many sprint stacks you need to reach it.
Byteglow has a great calculator to compute the number of stacks you need.
Remember that the more area you want to jump over, the more stacks are needed to jump over just one zone.
The rule of thumb I usually follow is to aim lower than 6000 stacks, after that it's taking too much time.
3000 to 5000 stacks are usually the higher range you want to aim at.
Under 3000 is usually fairly easy.


Having too little to no stacks will slow down the farm terribly.
For perfect jumps:
    Overstacking does not matter in the slightest.
    Understacking prevents you to jump to your stacking zone, preventing you to get enough stacks for the next run.

For non perfect jumps:
    Overstacking may jump passed your stacking zone, leaving you with little to no stacks.
    Understacking allows you to ensure that you will be at least some kind of fast, and walk the lasts areas as expected.

First, do not concern yourself with damages.
Only look at how much health your \briv needs to get enough stacks.

To improve survivability:
\begin{itemize}
    \item Mirtt's \textbf{Perk's Up} perk, that increases tank's healing.
    \item More health. there are multiple ways of increasing health.
        \begin{itemize}
            \item Most tanks have a health share ability.
            Moreover many tanks have a health scaling item.
            Therefore more tanks, more health.
            To have a tank provide health however, it requires to have some item level in their health item.
            This is why we often choose Evergreen champions that continuously scale up.
            Particularly, \dragonbait, \tyril and \nayeli.
            \tyril and \nayeli need to be in right specialization.
            \item Vjara and Zariel perks increase health moderately.
            \item Champion feats, particularly tanks.
            Careful, overwhelmed feats are better for \briv.
        \end{itemize}
    \item Health potions.
    Beware of using Health Potions.
    You will not be able to sustain them for a long (Section \ref{sec:potionSustain}).
    Note that the more tanks are used, the more health the potions are bringing.
    \item \baeloth's djinn ability gives 30 second of invincibility upon death.
\end{itemize}



\subsubsection{Step Four: Tune Your BUD}
\label{sec:stepFour}

Once you know at what time your stacks are high enough, you can prepare your BUD.
You should use a trigger champion that buffs your BUD setter.

\paragraph{Choosing a trigger champion.}

I identify 3 kinds of possible triggers
\begin{description}
    \item[Enemy count] There are a few champions that give damage boost depending on available enemy.
    For example, \shandie's explosive arrow.
    This gives a boost that may be enough.
    I found it to be the less consistent, and I usually try to \textbf{not} affect my BUD setter with explosive arrow.
    Moreover ranges enemies are scattered.
    Therefore the number of enemy may not be consistent.
    
    \item[Enrage stack count]
    After you get a 100 enemies in a non boss area, an enrage count starts.
    This is similar to bosses enrage count.
    Multiple tank champions buff the damages of champions depending on the number of enrage stacks.
    This is the case of \dragonbait, \tyril and \nayeli.
    Each stack of enrage gives a little bit more buff.
    
    \item[\briv death] At some point \briv should die due to the number of enrage stacks.
    \baeloth's is the only one that has worked so far (if someone wants to try out \desmond, give me feedback?).
    When \briv dies and \baeloth's Djinn trigger, a big amount of damage boost that should allow you to clear the area.
\end{description}

The trigger champion is often chosen by the reset level, and the damage \briv's is under ect.
For example, it's harder to use \briv death as a trigger at lower levels.
Similarly, using enemy count at high levels may not be appropriate because you need more stacks than the hundred enemy cap can provide.

\paragraph{Choosing a BUD setter}
BUD setter may be any character.
I even used \nerys at some point !

One property that I personally like in my BUD setters is to get amplify the trigger.
This is the case of \artemis, \arkhan (untested) or \blackViper (untested).
Personnally I used \artemis in every set up I do.
This is because he observe the trigger over multiple champions, and therefore benefits from an amplified effect.
However the jeweled power ability increase your next auto attack upon killing a unit.
Therefore your first attack with \artemis on the stacking zone should never kill a unit.

\paragraph{Additional ways of tweaking the damages.}

\begin{description}
    \item[Blessing and Patron.] I usually do not deal with them, because I always forget to put them back when doing something else.
    \item[Modron core nodes.] this gives pretty big boosts, even more so now that overcharged is a thing;
    \item[Feats.] Feats are pretty minor compared with Modron core nodes, I usually keep them for final tweaks.
    I also try to not touch them too much, to allow different setups.
    \item[Support champions] duh.
\end{description}


\subsubsection{Step Five: Monitor}
\label{sec:stepFive}

It worked once !\newline
It will always work !\newline

No.
Things are often inconsistent.
We usually have to watch a bit to see how the farm is behaving.

Things that may add inconsistency that we identified are:
\begin{description}
    \item[Lag.]
    The less champions and clickers you have, the less lag you are subject to.
    Less lag, more farm.
    More lag less farm, and inconsistencies.
    \item[Distractions.] Using a 6th clicker to kill distractions may give unexpected inputs of golds, which would give too much click damages.
    \item[Extra quest rewards.] Level 15th node in the speed core may give different gold inputs by ending some levels early.
    This is often not a problem for perfect jumps.
\end{description}


\subsubsection{Non-Perfect Jumps Particularity}
\label{sec:nonPerfectJumps}

\subsection{Choosing an Adventure}

The first question we get asked is which adventure to target ?
How do I choose my adventure ?
If you are only interested in existing results, I provide a list we compiled with their click wall, stacking zones and other relevant information.
If you want to know how to choose an adventure, it's after the list !


\begin{table}[ht!]
\caption{Adventures for Modron \briv gem farming.
The general adventures are adventures that can be chosen for any kind of jump chances, perfect or not.
Particularly, we currently do not have a good map for 6 or 8 jumps \briv .}
\label{tbl:adventures}
\begin{small}
\begin{tabular}{ l | l c c c }
\toprule
\textbf{Jumps} & \textbf{Campaign} & \textbf{Adventure} & \textbf{Click Wall} & \textbf{Stacking Zone}\\
\midrule
General & Sword Coast          & Supply Run & &\\
        &                      & Terror In the Dark & &\\
        & Descent Into Avernus & todo & &\\
\separatingLine
3       &  IceWind Dale & The Everlasting Rime (ER) &  &\\
\separatingLine
4       &  IceWind Dale & The Everlasting Rime (ER) &  & z41\\
        &  IceWind Dale & Tall Tales (TT) &  & z41\\
\separatingLine
7       &  IceWind Dale & The Everlasting Rime (ER) &  & z97\\
\separatingLine
9       &  IceWind Dale & Tall Tales (TT) &  & z41\\
\bottomrule
\end{tabular}
\end{small}
\end{table}


\subsubsection{3 Jumps}

\subsubsection{4 Jumps}

\subsubsection{7 Jumps}

\subsubsection{9 Jumps}

Back to the great Tall Tale adventure

\subsubsection{Specific Adventure Set Up}


\section{Frequently Asked Questions}
\subsection{Dealing With Too High Favors}
\subsection{I Want to Set Up On a Specific Map.}



\subsection{Getting Perfect Jump Chances}

How to get perfect jump chances ?
The newly released feat capping the effect at perfect 4 jumps is not the only way.
In the old days, and for people that do not target 4 jumps, we have to use up contracts one by one at the end.
For more than 10000 levels perfect jumps, it's cannot really see the precise item levels value in game.
You may have to use an external tool to view it in plain text such as Byteglow (ref).
And still go one by one once you get close to the expected value.

\subsection{Why Does Byteglow Tells Me I need That Much Item Level}

\paragraph{One by one}

Using up blacksmith contracts should be spread (almost) evenly between all items.
Therefore to go from one cap to the other, we have to get enough blacksmith contracts to get \textbf{all} items to the level that you aim for.

\paragraph{This is approximate !}
Beware !
All item level assignation are approximately evenly spread between item.
The key word being approximately.
You may need less, you may need more.
I personally tend to always take a bit more than I need, just in case.

\subsection{High Favor Farming}
\subsection{Buying and Opening Chests}
\subsection{Potion usage}
\label{sec:potionSustain}

\subsubsection{Potion Sustain}


\section{Asking for help}
//Todo, template to ask for help.



\end{document}
